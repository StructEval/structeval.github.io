% \begin{table*}[!h]
% \small
% \centering
% \begin{tabular}{|p{\textwidth}|}
% \hline
% \textbf{VQA Prompt Template} \\
% \hline
% You are given an image and a list of question-answer pairs. \\
% For each pair, verify if the image content supports the expected answer based on the corresponding question. \\
% Base your judgment solely on the visual content of the provided image, and the question. \\
% Do not use any external information or common-sense reasoning beyond what is visible. \\
% Respond with a JSON object mapping each question number to true or false (e.g., \{"1": true, "2": false\}). \\
% If the image is unclear or does not contain enough information to answer, use \texttt{null} for that question. \\
% Here are the question-answer pairs: \{qa\_list\} \\
% \hline
% \end{tabular}
% \caption{Prompte template used for VQA evaluation. We use GPT-4.1-mini in the benchmark evaluation}
% \label{tab:prompt_template}
% \end{table*}

% \begin{center}
% \begin{tcolorbox}[
%   colback=white,
%   colframe=gray!70,
%   title={\small\bfseries VQA Prompt Template},
%   colbacktitle=gray!10,
%   coltitle=black,
%   fontupper=\small,
%   enhanced,
%   left=2mm,
%   right=2mm,
%   boxrule=0.4pt,
%   arc=1mm
% ]

% You are given an image and a list of question-answer pairs. \\
% \begin{itemize}[nosep,leftmargin=*]
%   \item For each pair, verify if the image content supports the expected answer based on the corresponding question.\\
%   \item Base your judgment solely on the visual content of the provided image, and the question.\\
%   \item Do not use any external information or common-sense reasoning beyond what is visible.\\
%   \item Respond with a JSON object mapping each question number to true or false (e.g., \{"1": true, "2": false\}).\\
%   \item If the image is unclear or does not contain enough information to answer, use \texttt{null} for that question.\\
% \end{itemize}

% Here are the question-answer pairs: \{qa\_list\}
% \end{tcolorbox}
% \vspace{-8pt}
% \captionof{figure}{Prompt template used for VQA evaluation. We use GPT-4.1-mini in the benchmark evaluation.}
% \label{fig:vqa_prompt_template}
% \end{center}

% Then, create a figure environment to contain everything
\begin{figure}[!t]
    \centering
    \begin{tcolorbox}[
      colback=white,
      colframe=gray!70,
      title={\small\bfseries VQA Prompt Template},
      colbacktitle=gray!10,
      coltitle=black,
      fontupper=\small,
      enhanced,
      left=2mm,
      right=2mm,
      boxrule=0.4pt,
      arc=1mm
    ]
    You are given an image and a list of question-answer pairs. \\
    \begin{itemize}[nosep,leftmargin=*]
      \item For each pair, verify if the image content supports the expected answer based on the corresponding question.\\
      \item Base your judgment solely on the visual content of the provided image, and the question.\\
      \item Do not use any external information or common-sense reasoning beyond what is visible.\\
      \item Respond with a JSON object mapping each question number to true or false (e.g., \{"1": true, "2": false\}).\\
      \item If the image is unclear or does not contain enough information to answer, use \texttt{null} for that question.\\
    \end{itemize}
    Here are the question-answer pairs: \{qa\_list\}
    \end{tcolorbox}
    \caption{Prompt template used for VQA evaluation. We use GPT-4.1-mini in the benchmark evaluation.}
    \label{fig:vqa_prompt_template}
\end{figure}