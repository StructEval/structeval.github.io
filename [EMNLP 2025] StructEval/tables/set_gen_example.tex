% \begin{table*}[!h]
% \small
% \centering
% \begin{tabular}{|p{0.6\textwidth}|p{0.3\textwidth}|}
% \hline
% \textbf{StructEval-T Question} & \textbf{KeyWords} \\
% \hline
% Please output JSON code.

% Task:

% Summarize metadata about a fictional scientific article.

% Feature Requirements:

% 1. The top-level field "title" holds the article title as a string.

% 2. The field "authors" is a list containing exactly 2 items.

% 3. Each item in "authors" contains "name" (string) and "affiliation" (string).

% 4. The "publication.year" field gives the year as an integer.

% 5. The "keywords" field is a list of strings. & 
% title

% authors[0].name

% authors[1].affiliation

% publication.year

% keywords[2] \\
% \hline
% \end{tabular}
% \caption{Example question and key words of the StructEval-T generation task}
% \label{tab:set_gen_example}
% \end{table*}

% \begin{tcolorbox}[
%   colback=white,
%   colframe=gray!70,
%   title={\small\bfseries StructEval-T Question, KeyWords},
%   colbacktitle=gray!10,
%   coltitle=black,
%   fontupper=\small,
%   enhanced,
%   left=2mm,
%   right=2mm,
%   boxrule=0.4pt,
%   arc=1mm
% ]
% Please output \texttt{JSON} code.\\[4pt]
% \textbf{Task:}\\[2pt]
% Summarize metadata about a fictional scientific article.

% \vspace{6pt}
% \textbf{Feature Requirements:}\\[-5pt]
% \begin{enumerate}[nosep,label=\arabic*.,leftmargin=*]
%   \item Top-level field \texttt{"title"} is a string containing the article title.
%   \item Field \texttt{"authors"} is a list of exactly two items.
%   \item Each element of \texttt{"authors"} contains \texttt{"name"} (string) and \texttt{"affiliation"} (string).
%   \item Field \texttt{"publication.year"} is an integer.
%   \item Field \texttt{"keywords"} is a list of strings.
% \end{enumerate}

% \vspace{6pt}
% \hdashrule[0.5ex]{\linewidth}{0.5pt}{2pt 2pt}
% \vspace{-5pt}

% \textbf{Keywords:}\\[-5pt]
% \begin{itemize}[nosep,leftmargin=*]
%   \item \texttt{title}
%   \item \texttt{authors[0].name}
%   \item \texttt{authors[1].affiliation}
%   \item \texttt{publication.year}
%   \item \texttt{keywords[2]}
% \end{itemize}
% \end{tcolorbox}
% \captionof{figure}{Example question and key words of the StructEval-T generation task}
% \label{fig:set_gen_example}

\begin{figure}[!th]
    \begin{tcolorbox}[
      colback=white,
      colframe=gray!70,
      title={\small\bfseries StructEval-T Question, KeyWords},
      colbacktitle=gray!10,
      coltitle=black,
      fontupper=\small,
      enhanced,
      left=2mm,
      right=2mm,
      boxrule=0.4pt,
      arc=1mm
    ]
    Please output \texttt{JSON} code.\\[4pt]
    \textbf{Task:}\\[2pt]
    Summarize metadata about a fictional scientific article.
    \vspace{6pt}
    \textbf{Feature Requirements:}\\[-5pt]
    \begin{enumerate}[nosep,label=\arabic*.,leftmargin=*]
      \item Top-level field \texttt{"title"} is a string containing the article title.
      \item Field \texttt{"authors"} is a list of exactly two items.
      \item Each element of \texttt{"authors"} contains \texttt{"name"} (string) and \texttt{"affiliation"} (string).
      \item Field \texttt{"publication.year"} is an integer.
      \item Field \texttt{"keywords"} is a list of strings.
    \end{enumerate}
    \vspace{6pt}
    \hdashrule[0.5ex]{\linewidth}{0.5pt}{2pt 2pt}
    \vspace{-5pt}
    \textbf{Keywords:}\\[-5pt]
    \begin{itemize}[nosep,leftmargin=*]
      \item \texttt{title}
      \item \texttt{authors[0].name}
      \item \texttt{authors[1].affiliation}
      \item \texttt{publication.year}
      \item \texttt{keywords[2]}
    \end{itemize}
    \end{tcolorbox}
    \caption{Example question and key words of the StructEval-T generation task}
    \label{fig:set_gen_example}
\end{figure}
